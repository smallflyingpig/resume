% !TEX TS-program = xelatex
% !TEX encoding = UTF-8 Unicode
% !Mode:: "TeX:UTF-8"

\documentclass{resume}
\usepackage{zh_CN-Adobefonts_external} % Simplified Chinese Support using external fonts (./fonts/zh_CN-Adobe/)
%\usepackage{zh_CN-Adobefonts_internal} % Simplified Chinese Support using system fonts
\usepackage{linespacing_fix} % disable extra space before next section
\usepackage{cite}
\usepackage{hyperref}
\hypersetup{
    colorlinks=true,
    linkcolor=black,
    urlcolor=blue,
    }

\begin{document}
\pagenumbering{gobble} % suppress displaying page number

\name{黎吉国}

% {E-mail}{mobilephone}{homepage}
% be careful of _ in emaill address
\contactInfo{(+86)15652370966}{jgli@pku.edu.cn}{多媒体算法工程师}{\href{https://github.com/smallflyingpig}{GitHub}~\textbf{|}~\href{https://scholar.google.com/citations?user=NfQeyQ4AAAAJ&hl=zh-CN}{Google Scholar}}
% {E-mail}{mobilephone}
% keep the last empty braces!
%\contactInfo{xxx@yuanbin.me}{(+86) 131-221-87xxx}{}
 
\section{个人简介}
中国科学院计算技术研究所博士在读, 现于北京大学数字媒体所视频编解码国家工程实验室培养, 研究兴趣包括\textbf{图像处理与生成,跨媒体分析,虚拟现实}等. 求职意向为\textbf{图像/视频生成}相关的多媒体算法岗位.

% \section{\faGraduationCap\ 教育背景}
\section{教育背景}
\datedsubsection{\textbf{中国科学院大学计算技术研究所},计算机应用技术,\textit{在读博士研究生}}{2016.9 - 至今}
\begin{itemize}
  \item 北京大学数字视频编解码技术国家工程实验室培养,导师为\href{http://www.jdl.ac.cn/htm-gaowen/}{高文教授},\href{https://scholar.google.com/citations?user=y3YqlaUAAAAJ&hl=zh-CN}{马思伟教授}.
  \item 研究方向为图像处理与生成,跨媒体分析,虚拟现实等.
  \item 获中国科学院大学学业奖学金, \href{http://www.avs.org.cn/avs_award/2017.asp}{AVS产业创新奖}, 预计2022年毕业.
\end{itemize}

\datedsubsection{\textbf{华中科技大学},自动化,\textit{工学学士}}{2012.9 - 2016.6}
\begin{itemize}
  \item 华中科技大学启明学院特优生,导师为\href{https://scholar.google.com.sg/citations?user=396o2BAAAAAJ&hl=zh-CN}{曹治国教授}.
  \item 智能车团队成员, 获全国大学生智能汽车大赛华南赛区一等奖.
  \item 就读期间获国家奖学金, 国家励志奖学金.
\end{itemize}


% \datedsubsection{\textbf{荷兰 莱顿大学},计算机科学与技术,\textit{国家留学基金委公派交换生}}{2015.3 - 2015.5}
% \ 2014年中国政府奖学金(\textit{http://www.csc.edu.cn/}),DID-ACTE项目交换生(\textit{http://did-acte.org/})

\section{学术成果\&学术服务}
\begin{itemize}
  \item Li Jiguo, Jia Chuanmin, Zhang Xinfeng, Ma Siwei, Gao Wen. Cross Modal Compression: Towards Human-comprehensible Semantic Compression[C]. The 29th ACM International Conference on Multimedia. (Submited, CCF A, 一作)
  \item Li Jiguo, Zhang Xinfeng, Jia Chuanmin, Xu Jizheng, Zhang Li, Wang Yue, Ma Siwei, Gao Wen. Direct speech-to-image translation[J]. IEEE Journal of Selected Topics in Signal Processing, 2020, 14(3): 517-529. (Published, IF 4.981, 一作, \href{https://smallflyingpig.github.io/speech-to-image/main}{Homepage})
  \item Li Jiguo, Zhang Xinfeng, Xu Jizheng, Ma Siwei, Gao Wen. Learning to Fool the Speaker Recognition[J]. ACM Transactions on Multimedia Computing, Communications, and Applications. (Accepted, CCF B, IF 3.275, 一作, \href{https://smallflyingpig.github.io/speaker-recognition-attacker/main}{Homepage})
  \item Li Jiguo, Zhang Xinfeng, Jia Chuanmin, Xu Jizheng, Zhang Li, Wang Yue, Ma Siwei, Gao Wen. Universal adversarial perturbations generative network for speaker recognition[C]//2020 IEEE International Conference on Multimedia and Expo (ICME). IEEE, 2020: 1-6. (Published, CCF B, 一作, \href{https://smallflyingpig.github.io/UAPs_for_speaker_recognition/main}{Homepage})
  \item Li Jiguo, Zhang Xinfeng, Xu Jizheng, Zhang Li, Wang Yue, Ma Siwei, Gao Wen. Learning to fool the speaker recognition[C]//ICASSP 2020-2020 IEEE International Conference on Acoustics, Speech and Signal Processing (ICASSP). IEEE, 2020: 2937-2941. (Published, CCF B, 一作, \href{https://smallflyingpig.github.io/speaker-recognition-attacker/main}{Homepage})
  \item 黎吉国, 王悦, 张新峰, 马思伟, 高文. 一种鱼眼视频全景拼接中的亮度补偿算法[J]. 中国科学: 信息科学, 2018. (Published, CCF A, 一作)
  \item 长期担任TCSVT审稿人, 受邀担任ICME2020\&ICME2021审稿人.
\end{itemize}
% \section{\faCogs\ IT 技能}

\section{实习经历}
\datedsubsection{\textbf{字节跳动 | Bytedance}, 多媒体算法实习生}{2018.06-2020.09}
\begin{itemize}
   \item 视频架构-多媒体实验室的research intern, mentor为\href{https://scholar.google.com/citations?user=x4iWZ7wAAAAJ&hl=en}{许继征博士}/\href{https://scholar.google.com/citations?hl=en&user=RQD_dqYAAAAJ}{施澍博士}.
  \item 在许继征博士指导下发表/接收期刊论文两篇,会议论文两篇,申请\href{http://www.soopat.com/Home/Result?Sort=&View=&Columns=&Valid=&Embed=&Db=&Ids=&FolderIds=&FolderId=&ImportPatentIndex=&Filter=&SearchWord=%E9%BB%8E%E5%90%89%E5%9B%BD&FMZL=Y&SYXX=Y&WGZL=Y&FMSQ=Y}{发明专利}三项.
  \item 在施澍博士指导下设计开发用于自动测量云游戏延时的算法和系统,申请\href{http://www.soopat.com/Patent/202110172567}{发明专利}一项.
\end{itemize}

\datedsubsection{\textbf{维境视讯 | VSCENE}, 图像算法实习生}{2016.09-2018.06}
\begin{itemize}
  \item 维境视讯成立于2016年, 于2019年被并购.
  \item 兼职图像算法实习生, 在\href{https://dblp.uni-trier.de/pid/33/4822-32.html}{王悦博士}的指导下, 负责全景直播系统中的视频实时拼接算法的研究和实现, 相关技术和实现应用于维境视讯的全景直播系统中.
  \item 相关技术发表\href{https://kns.cnki.net/kcms/detail/detail.aspx?dbcode=CJFD&dbname=CJFDLAST2018&filename=PZKX201803003&v=tPaKZsLxkg0cUv5TgA0vLc%25mmd2B%25mmd2FYm0D6CGdqsd20qB21aAFkQVtxA1pTOQHC3je4lKJ}{期刊论文}一篇, 申请\href{http://www.soopat.com/Patent/201711100069}{发明专利}一项, 相关系统曾获得\href{http://www.avs.org.cn/avs_award/2017.asp}{AVS产业创新奖}.
\end{itemize}

\section{技术能力}
% increase linespacing [parsep=0.5ex]
\begin{itemize}[parsep=0.2ex]
  \item Python, C++, Pytorch, Linux, Cuda, \LaTeX, Git, CET 6
\end{itemize}

\section{项目经历}
\datedsubsection{\textbf{跨模态压缩} | Python | Pytorch}{2020.01-2020.09}
\begin{itemize}
  \item 跨模态压缩: 通过设计一个人可理解的, 语义紧凑的压缩域, 将高冗余的视觉数据压缩至低冗余的压缩域, 同时保持语义的保真和可恢复性. 
  \item 跨模态压缩的一个实例为图像-文本-图像的跨模态压缩. 其中文本域为语义紧凑的压缩域, 同时文本也是一种人可以理解的语义表示. 利用图像描述模型得到图像在文本域的表示, 利用图像生成模型可以将文本表示恢复为语义一致的图像. 同时文本可以利用Huffman编码进行无损压缩. 
  \item 图像-文本-图像的跨模态压缩框架在IS, FID等指标指标下的RD性能超过了JPEG和JPEG2000.
  \item 相关成果投稿至CCF A类国际会议ACM MM 2021. 相关技术已申请\href{http://www.soopat.com/Patent/202010604773}{发明专利}.
\end{itemize}
\datedsubsection{\textbf{云游戏时延自动测量系统} | Python}{2020.01-2020.09}
\begin{itemize}
  \item 云游戏的延时是影响用户体验的最重要因素之一,但是当前延时的测量依赖大量的人工, 费时费力.
  \item 通过后台数据实时收集鼠键控制输入和屏幕的变化,分析控制输入和屏幕内容变化之间的对应关系,筛选出响应显著的控制输入,从而完成自动化的延时测量。
  \item 相关原型系统可运行于后台, 不影响用户操作.
  \item 相关系统已申请\href{http://www.soopat.com/Patent/202110172567}{发明专利}。
\end{itemize}
\datedsubsection{\textbf{空时域融合的声纹识别方法} | Pytorch | Pytorch}{2019.06-2019.12}
\begin{itemize}
  \item 传统的声纹识别先将声音数据变换为频谱, 基于频谱进行特征提取和声纹识别.
  \item 近期的声纹的识别和生成开始尝试直接在时域进行处理和生成.
  \item 本项目尝试将声音数据的频谱处理和时域的特征提取相结合, 通过融合时域和频域的信息, 得到了比单一域特征提取更好的声纹识别结果.
  \item 相关技术已申请\href{http://www.soopat.com/Patent/202010694750}{发明专利}.
\end{itemize}
\datedsubsection{\textbf{面向声纹识别的对抗攻击方法} | Python | Pytorch}{2019.06-2019.12}
\begin{itemize}
  \item 利用人耳对声音的感知特性: 人耳会重点关注有实际含义的声音而忽略无意义的噪声, 设计一种可以更好保持语音中因素信息的对抗攻击方法, 明显提升了针对语音识别对抗攻击的效率. 相关成果被语音顶级国际会议ICASSP2020接收. 相关技术已申请\href{http://www.soopat.com/Patent/202011024815}{发明专利}.
  \item 通过分析不同频带限制下的对抗扰动的攻击效率, 发现高频段的对抗扰动可以实现更高的攻击效率. 相关成果被ACM多媒体及其应用汇刊TOMM接收.
  \item 利用对抗训练的方式学习面向声纹识别的通用对抗噪声, 对抗攻击效果明显高于随机噪声, 相关成果被国际会议ICME2020接收.
\end{itemize}
\datedsubsection{\textbf{基于语音描述的图像生成} | Python | Pytorch}{2018.09-2019.06}
\begin{itemize}
  \item 探索直接从语音描述中生成图像, 而不借助中间文本表示的系统和压缩框架. 
  \item 利用预训练的ResNet提取图像特征, 设计基于CNN的音频编码器, 利用迁移学习训练音频编码器. 
  \item 设计基于GAN的生成网络, 以音频特征为条件生成图像, 利用多个判别器来提高生成图像的分辨率.
  \item 基于合成语音生成图像的效果, 与基于文本的图像生成, 在FID, IS等指标上接近.
  \item 相关成果整理为期刊论文发表于IEEE期刊JSTSP. 相关技术已申请\href{http://www.soopat.com/Patent/202010604773}{发明专利}.
\end{itemize}
\datedsubsection{\textbf{4K 全景直播系统} | C++ | Cuda}{2016.09-2018.06}
\begin{itemize}
  \item 863项目: 真三维视频关键技术研究与先导验证, 演示系统之一, 负责其中的视频实时拼接部分. 
  \item 利用视频采集卡实时或许6-8路视频流, 设计拼接算法将多路视频流逐帧实时拼接后显示或发送至推流服务器进行分发. 
  \item 利用GPU进行逐像素的融合(cuda实现),利用GPU流水线进行拼接-亮度补偿-融合的加速, 从而实现多路高清视频的实时拼接.
  \item 相关拼接算法可实时运行于带GPU的移动工作站(笔记本).
  \item 相关技术和算法应用于维境视讯有限公司全景直播系统中.
  \item 相关系统申请\href{http://www.soopat.com/Patent/201711100069}{发明专利}, 部分成果整理后发表于CCF A类中文期刊《中国科学:信息科学》.
\end{itemize}

\datedsubsection{\textbf{英文自动拼写矫正系统} | Python}{2017.05-2017.06}
\begin{itemize}
  \item 利用词库和单词之间的编辑距离,自动校正错误的单词拼写,
  \item \href{https://github.com/smallflyingpig/projects}{Github}开源后获得50+ forks.
\end{itemize}


%% Reference
%\newpage
%\bibliographystyle{IEEETran}
%\bibliography{mycite}
\end{document}
